\documentclass[../thesis]{subfiles}

\begin{document}

\chapter{Architecture}
Characterization and development of sensor arrays presents a broad
range of research challenges, not least of which relate to data
organization. A \gls{LIMS} adequate to the needs of our example
application must provide

This chapter abstractly describes the constituent components
of the software framework we have built for collaborative design,
execution, and analysis of experiments. When describing each
element, we document some of the phases of our iterative design
process that led to these decisions.

\section{Physical architecture}

\section{Network architecture}
% \begin{figure}
%   \includegraphics[width=\textwidth]{network-monolithic}
%   \caption{
%     Schematic of an example experimental apparatus for
%     characterizing an array of electrochemical gas sensors.
%     Inset: some commonly used stimulus waveforms for interrogating
%     electrochemical sensors.
%     \label{fig:EchemUseCase}
%   }
% \end{figure}

\subsection{Monolithic approach}

\subsection{Microservice-oriented}

\subsection{Switchboard service}

\section{Data model}

\subsection{entity models}
\subsection{dataset tabulation}



\section{Device control}
A core goal of our design is to enable researchers to incorporate
choreography of physical lab equipment into the executable workflows
they create. Interact with the variety of commercial and custom
hardware found in a typical experimental lab requires a flexible
approach, given that computer control interfaces and data formats for
scientific equipment are heterogeneous and very poorly
standardized. This section describes an approach for building a
modular library of device drivers which integrate with the rest of the
system while providing users with tools for extension and
customization.

\subsection{Instrument manager}
The instrument manager is a microservice responsible for detecting
connected devices, determining the appropriate device driver for
communicating with them, and presenting a unified interface to the
switchboard.

\subsection{Interface API}

\subsection{Enumeration}

\subsection{Protocol composition}



\section{User experience}



\section{Security}

\end{document}
