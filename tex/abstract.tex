\documentclass[../thesis]{subfiles}

\begin{document}

\singlespacing
\begin{abstract}
  Reproducible research has been recognized as a growing concern in
  most areas of science. To achieve widespread adoption of repeatable,
  transparent research practices, some commentators have identified a
  need for better software for authoring reproducible digital
  publications. Complicating this goal, scientific investigations
  increasingly involve interdisciplinary teams, sophisticated
  workflows for acquiring and analyzing data, and huge datasets that
  rely on considerable metadata to interpret. Computational scientists
  have begun to adopt tools for managing the complex histories of
  their data and procedures, but software which simultaneously allows
  researchers to specify experiments, remotely control equipment, and
  capture and organize data remains immature. This thesis demonstrates
  a software architecture for programmable remote control of custom
  and commercial lab equipment, automatic annotation and queryable
  storage of data sets, and provenance-aware specification of
  experiment and analysis procedures. The design consists of a suite
  of small, single-purpose software services which may be controlled
  remotely from a web browser, notably including a graphical
  programming tool, an abstraction layer for interfacing with
  commercial hardware and custom embedded systems, and a hybrid
  document/table database for persistent storage of annotated
  experimental data. The software implementation embraces modern web
  technologies and best practices to produce a modular,
  user-extensible framework that is well-suited for helping to
  integrate computer-controlled research labs with the emerging
  Internet of Things.
\end{abstract}
\doublespacing

\end{document}
