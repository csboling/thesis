\newglossaryentry{artifact}{
  name={artifact},
  description={a generic term for an entity of interest to a research
    project, especially a data set, source code of a program, or an
    experimental protocol specification}
}

\newglossaryentry{AMQPg}{
  name={AMQP},
  description={a network protocol for sharing data and computations
    between a cluster of connected computers. Implemented most prominently by
    RabbitMQ \cite{RabbitMQ}}
}
\newglossaryentry{AMQP}{
  type=\acronymtype,
  name={AMQP},
  description={Advanced Message Queueing Protocol},
  first={Advanced Message Queueing Protocol (AMQP)\glsadd{AMQPg}},
  see=[Glossary:]{AMQPg}
}

\newglossaryentry{APIg}{
  name={API},
  description={An application programming interface (API) is a
    publically exposed set of software functionality intended to be
    used for composing other applications. In a web programming
    context, sometimes a \gls{RESTAPI} is meant}
}
\newglossaryentry{API}{
  type=\acronymtype,
  name={API},
  description={application programming interface},
  first={application programming interface (API)\glsadd{APIg}},
  see=[Glossary:]{APIg}
}

\newglossaryentry{backend}{
  name={backend},
  description={the internal server-side logic of a web app, typically
    responsible for interacting with databases and performing
    intensive computations
  }
}

\newglossaryentry{BPMNg}{
  name={BPMN},
  description={Business Process Model Notation, a specification of a
    flowchart-like format for describing procedures found in business
    and manufacturing settings. See \cite{Allweyer:2010:BPM:1841147}}
}
\newglossaryentry{BPMN}{
  type=\acronymtype,
  name={BPMN},
  description={business process model notation},
  first={Business Process Model Notation (BPMN)\glsadd{BPMNg}},
  see=[Glossary:]{BPMNg}
}

\newglossaryentry{dataProvenance}{
  name={data provenance},
  description={A generalized term for tracking scientific data as it
    undergoes a sequence of transformations from raw data into
    a publication-ready figure}
}

\newglossaryentry{databaseSchema}{name={database schema},
  description={
a specification describing the structure of allowed
    database entries
}
}

\newglossaryentry{declarative}{name={declarative},
  description={
    a programming style which focuses on asserting relationships
    between software entities rather than describing the state
    transitions necessary to transform data
  }
}

\newglossaryentry{designPattern}{
  name={design pattern},
  description={a recurring, reusable approach for structuring
    software; a blueprint for how a particular programming problem may
    be solved
  }
}

\newglossaryentry{DOMg}{
  name={DOM},
  description={a term for the data structure used to represent
    components of a web page, namely the tree of HTML or XML tags and
    their associated properties}
}
\newglossaryentry{DOM}{
  type=\acronymtype,
  name={DOM},
  description={document object model},
  first={document object model (DOM)\glsadd{DOMg}},
  see=[Glossary:]{DOMg}
}

\newglossaryentry{ELNg}{
  name={ELN},
  description={An electronic lab notebook (ELN) is a software tool for
    helping researchers to chronicle their day-to-day investigations
    and results by composing rich-text documents which consolidate
    data, code, plots, and natural-language research questions and
    analysis}
}
\newglossaryentry{ELN}{
  type=\acronymtype,
  name={ELN},
  description={electronic lab notebook},
  first={electronic lab notebook (ELN)\glsadd{ELNg}},
  see=[Glossary:]{ELNg}
}

\newglossaryentry{frontend}{
  name={frontend},
  description={the user-facing portion of a web app, e.g. the
    graphical interface displayed by a browser}
}

\newglossaryentry{insilico}{
  name={in-silico},
  description={A designation applied to scientific endeavors which
    consist entirely of computer analysis of data, named in contrast
    with \textit{in-vivo} biological experiments}
}

\newglossaryentry{IoTg}{
  name={IoT},
  description={The Internet of Things describes a near-future network
    infrastructure characterized by unprecedented device-to-device
    communication and ubiquitous Internet-capable sensors and actuators}
}
\newglossaryentry{IoT}{
  type=\acronymtype,
  name={IoT},
  description={Internet of Things},
  first={Internet of Things (IoT)\glsadd{IoTg}},
  see=[Glossary:]{IoTg}
}

\newglossaryentry{JSONg}{
  name={JSON},
  description={
    a simple text format, originally native to JavaScript, for encoding hierarchical
    data structures containing fields of many different types
  }
}
\newglossaryentry{JSON}{ type=\acronymtype, name={JSON},
  description={JavaScript Object Notation}, first={JavaScript Object
    Notation (JSON)\glsadd{JSONg}}, see=[Glossary:]{JSONg} }

\newglossaryentry{lazyLoad}{
  name={lazy-load},
  description={
    a strategy where an application waits to load subcomponents until
    they are about to be used, decreasing the program's startup time
    and allowing it to depend on resources which may not be locatable
    until runtime information is available
  }
}

\newglossaryentry{LIMSg}{
  name={laboratory information management system (LIMS)},
  description={A bundle of software tools for coordinating the activities of
    researchers, tracking inventory and data sets, and describing and
    monitoring experimental processes in one or more laboratories}
}
\newglossaryentry{LIMS}{
  type=\acronymtype,
  name={LIMS},
  description={laboratory information management system},
  first={laboratory information management system
    (LIMS)\glsadd{LIMSg}},
  see=[Glossary:]{LIMSg}
}

\newglossaryentry{MEAN}{
  type=\acronymtype,
  name={MEAN},
  description={a web application software stack consisting of MongoDB
    (database), ExpressJS (web server middleware), AngularJS (client
    front-end), and NodeJS (network programming architecture)},
  first={MongoDB, ExpressJS, AngularJS, and NodeJS (MEAN)},
}

\newglossaryentry{microservice}{
  name={microservice},
  description={a small, single-purpose web application intended to
    communicate with a collection of other microservices}
}

\newglossaryentry{protocolStack}{
  name={protocol stack},
  description={a series of translation steps converting one
    communication protocol into another}
}

\newglossaryentry{provenance}{
  name={provenance},
  description={See \gls{dataProvenance}.}
}

\newglossaryentry{researchObject}{
  name={research object},
  description={a proposed type of rich electronic publication format for packaging
  data, executable procedures, and documentation in a single
  semantically-linked archive}
}

\newglossaryentry{RDFg}{
  name={RDF},
  description={a formalism for encoding graphs of semantic connections
  between entities via subject-verb-object triples which serves as the
base language level for the W3C's \gls{semanticWeb} standards}
}
\newglossaryentry{RDF}{
  type=\acronymtype,
  name={RDF},
  description={resource description framework},
  first={resource description framework (RDF)\glsadd{RDFg}},
  see=[Glossary:]{RDFg}
}

\newglossaryentry{RESTg}{
  name={REST},
  description={a software architecture where clients interact with
    servers by navigating a sequence of states or resources, each
    associated with a particular URI}
}
\newglossaryentry{REST}{
  type=\acronymtype,
  name={REST},
  description={representational state transfer},
  first={representational state transfer (REST)\glsadd{RESTg}},
  see=[Glossary:]{RESTg}
}

\newglossaryentry{RESTAPI}{
  name={REST API},
  description={An API adhering to Representational State Transfer
    (REST) principles. REST APIs are endpoints for issuing control and
  data commands over an HTTP interface, allowing web servers to
  expose functionality over the internet in a client-agnostic fashion}
}

\newglossaryentry{RPCg}{
  name={RPC},
  description={a programming abstraction where a sequence of network
    transactions is thought of as one machine remotely invoking a
    subroutine over the network, receiving its return value as a response}
}
\newglossaryentry{RPC}{
  type=\acronymtype,
  name={RPC},
  description={remote procedure call},
  first={remote procedure call (RPC)\glsadd{RPCg}},
  see=[Glossary:]{RPCg}
}

\newglossaryentry{scientificPriority}{
  name={scientific priority},
  description={credit for being the first to publish or describe an
    invention or discovery}
}

\newglossaryentry{semanticWeb}{
  name={semantic web},
  description={an approach to knowledge management where Internet
    resources are annotated with groups of hyperlinks describing their
    relationships to other resources}
}

\newglossaryentry{SQLg}{
  name={SQL},
  description={a standardized language for accessing and managing
    databases using sets of search criteria}
}
\newglossaryentry{SQL}{
  type=\acronymtype,
  name={SQL},
  description={Structured Query Language},
  first={Structured Query Language (SQL)\glsadd{SQLg}},
  see=[Glossary:]{SQLg}
}

\newglossaryentry{switchboard}{
  name={switchboard},
  description={a \gls{microservice} responsible for determining which
    other microservices are active and making them available at
    appropriate \glspl{URI}
  }
}

\newglossaryentry{thinClient}{
  name={thin client},
  description={
    a hardware or software component which acts as a lightweight portal
    connecting users to server-side functionality, involving little or
    no client-side software to use
  }
}

\newglossaryentry{UIg}{
  name={UI},
  description={the portion of a software application concerned with
    accepting input from the user and producing output; often
    synonymous with graphical user interface (GUI)}
}
\newglossaryentry{UI}{
  type=\acronymtype,
  name={UI},
  description={user interface},
  first={user interface (UI)\glsadd{UIg}},
  see=[Glossary:]{UIg}
}

\newglossaryentry{URIg}{
  name={URI},
  description={a text string uniquely identifying an Internet resource}
}
\newglossaryentry{URI}{
  type=\acronymtype,
  name={URI},
  description={uniform resource identifier},
  first={uniform resource identifier (URI)\glsadd{URIg}},
  see=[Glossary:]{URIg}
}

\newglossaryentry{VCSg}{
  name={VCS},
  description={A set of software features related to tracking file
    revisions and allowing authors to revert files to previous
    states}
}
\newglossaryentry{VCS}{
  type=\acronymtype,
  name={VCS},
  description={version control system},
  first={version control system (VCS)\glsadd{VCSg}},
  see=[Glossary:]{VCSg}
}

\newglossaryentry{VISAg}{
  name={VISA},
  description={an industry standard specification for the
    communication interface that a scientific testing or measurement
    instrument should provide}
}
\newglossaryentry{VISA}{
  type=\acronymtype,
  name={VISA},
  description={Virtual Instrument Software Architecture},
  first={Virtual Instrument Software Architecture (VISA)\glsadd{VISAg}},
  see=[Glossary:]{VISAg}
}

\newglossaryentry{WAMPg}{
  name={WAMP},
  description={a high-level application protocol built on top of
    WebSockets for allowing heterogeneous services to communicate via
    remote procedure calls and publish/subscribe event streams}
}
\newglossaryentry{WAMP}{
  type=\acronymtype,
  name={WAMP},
  description={Web Application Messaging Protocol},
  first={Web Application Messaging Protocol (WAMP)\glsadd{WAMPg}},
  see=[Glossary:]{WAMPg}
}

\newglossaryentry{WMSg}{
  name={WMS},
  description={A software package for creating and composing directed
    graphs of process phases and/or dependencies.}
}
\newglossaryentry{WMS}{
  type=\acronymtype,
  name={WMS},
  description={workflow management system},
  first={workflow management system (WMS)\glsadd{WMSg}},
  see=[Glossary:]{WMSg}
}


%%% Local Variables:
%%% mode: latex
%%% TeX-master: "thesis"
%%% End:
