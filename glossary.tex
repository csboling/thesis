\newglossaryentry{APIg}{
  name={API},
  description={An application programming interface (API) is a
    publically exposed set of software functionality intended to be
    used for composing other applications. In a web programming
    context, sometimes a \gls{RESTAPI} is meant}
}
\newglossaryentry{API}{
  type=\acronymtype,
  name={API},
  description={application programming interface},
  first={application programming interface (API)\glsadd{APIg}},
  see=[Glossary:]{APIg}
}

\newglossaryentry{BPMNg}{
  name={BPMN},
  description={Business Process Model Notation, a specification of a
    flowchart-like format for describing procedures found in business
    and manufacturing settings. See \cite{Allweyer:2010:BPM:1841147}}
}
\newglossaryentry{BPMN}{
  type=\acronymtype,
  name={BPMN},
  description={business process model notation},
  first={Business Process Model Notation (BPMN)\glsadd{BPMNg}},
  see=[Glossary:]{BPMNg}
}

\newglossaryentry{dataProvenance}{
  name={data provenance},
  description={A generalized term for tracking scientific data as it
    undergoes a sequence of transformations from raw data into
    a publication-ready figure}
}

\newglossaryentry{ELNg}{
  name={ELN},
  description={An electronic lab notebook (ELN) is a software tool for
    helping researchers to chronicle their day-to-day investigations
    and results by composing rich-text documents which consolidate
    data, code, plots, and natural-language research questions and
    analysis}
}
\newglossaryentry{ELN}{
  type=\acronymtype,
  name={ELN},
  description={electronic lab notebook},
  first={electronic lab notebook (ELN)\glsadd{ELNg}},
  see=[Glossary:]{ELNg}
}

\newglossaryentry{insilico}{
  name={in-silico},
  description={A designation applied to scientific endeavors which
    consist entirely of computer analysis of data, named in contrast
    with \textit{in-vivo} biological experiments}
}

\newglossaryentry{IoTg}{
  name={IoT},
  description={The Internet of Things describes a near-future network
    infrastructure characterized by unprecedented device-to-device
    communication and ubiquitous Internet-capable sensors and actuators}
}
\newglossaryentry{IoT}{
  type=\acronymtype,
  name={IoT},
  description={Internet of Things},
  first={Internet of Things (IoT)\glsadd{IoTg}},
  see=[Glossary:]{IoTg}
}

\newglossaryentry{LIMSg}{
  name={laboratory information management system (LIMS)},
  description={A bundle of software tools for coordinating the activities of
    researchers, tracking inventory and data sets, and describing and
    monitoring experimental processes in one or more laboratories}
}
\newglossaryentry{LIMS}{
  type=\acronymtype,
  name={LIMS},
  description={laboratory information management system},
  first={laboratory information management system
    (LIMS)\glsadd{LIMSg}},
  see=[Glossary:]{LIMSg}
}

\newglossaryentry{RESTAPIg}{
  name={REST API},
  description={An API adhering to Representational State Transfer
    (REST) principles. REST APIs are endpoints for issuing control and
  data commands over a \gls{HTTP} interface, allowing web servers to
  expose functionality over the internet in a client-agnostic fashion}
}
\newglossaryentry{RESTAPI}{
  type=\acronymtype,
  name={REST API},
  description={representational state transfer application programming interface},
  first={REST API\glsadd{APIg}},
  see=[Glossary:]{RESTAPIg}
}

\newglossaryentry{VCSg}{
  name={VCS},
  description={A set of software features related to tracking file
    revisions and allowing authors to revert files to previous
    states}
}
\newglossaryentry{VCS}{
  type=\acronymtype,
  name={VCS},
  description={version control system},
  first={version control system (VCS)\glsadd{VCSg}},
  see=[Glossary:]{VCSg}
}

\newglossaryentry{workflowMgmt}{
  name={workflow management system},
  description={A software package for creating and composing directed
    graphs of process phases and/or dependencies.}
}

%%% Local Variables:
%%% mode: latex
%%% TeX-master: "thesis"
%%% End:
